\chapter{Motivation}
\label{chap:moti}
Medien haben einen großen Einfluss auf unsere heutige Gesellschaft und auf die Demokratie.
So wird die öffentliche Berichterstattung teilweise als \enquote{Vierte Gewalt} bezeichnet( 
in Anlehnung an die drei Staatsgewalten Legislative, Exekutive und Judikative). Diese hat 
zwar keine direkte Gewalt zur Änderung der Politik, kann diese aber durch ihre öffentliche 
Wirkung beeinflussen. Mit der Entstehung jedes neuen Mediums im letzten Jahrhundert kamen daher 
Bedenken auf, dass diese demokratiegefährdend sein könnten~\cite{hunt_fake}.
Bedenken bei der Einführung von Fernsehnachrichten waren beispielsweise, dass vorallem fotogene Menschen
einen Wahlkampf gewinnen würden. \\
Mit Beginn der Berichterstattung von Zeitungen im Internet,
kamen Bedenken zu Filterblasen auf, in denen sich lediglich Menschen gleicher 
Gesinnung austauschen und somit von entgegengesetzten Meinungen abgekapselt werden.\\
Heutzutage sind es soziale Medien, in denen \enquote{Fake News} eine Gefahr für unsere Demokratie 
darstellen. Mit Fake News werden in dieser Arbeit Nachrichten bezeichnet, die mit Falschinformationen
gezielt Meinungen (meist politische) beeinflussen sollen. Dieser Begriff gewann während der 
US-Präsidentschaftswahl 2016 an Bedeutung und wurde im selbigen Jahr zum Anglizismus des Jahres gewählt~\cite{nzz}.
Problematisch ist dabei vorallem, dass jeder Informationen verbreiten 
kann, ohne dabei die Überprüfung eines Dritten zu benötigen, diese aber gleichwertig
zu redaktionell geprüften Zeitungen in den sozialen Medien dargestellt werden. Zusätzlich 
wird es einfacher überzeugende Falschinformationen zu erzeugen. Beispielsweise ist es möglich
sogenannte \enquote{Deepfakes} zu erzeugen, bei denen Gesichter in Videos ausgetauscht werden.
Diese können aktuell jedoch mit CNN (convolutional neural networks) erkannt werden~\cite{deepfake}.
Es wird in Zukunft also nötig sein Methoden zu entwickeln, mit denen sich Fake News sicher erkennen lassen.\\
Da Methoden des maschinellen Lernens und der quantitativen Sprachwissenschaft bereits zur Klassifikation
von Texten eingesetzt werden, ist die Fragestellung dieser Arbeit, ob sich solche Methoden zur Erkennung 
von Fake News eignen. Die Frage ist hier auf einen zeitlich und inhaltlich 
limitierten Datensatz begrenzt, welcher mit einem einfachen Sprachmodell quantifiziert wird.
Zur Klassifikation wird dabei ein DNN (deep neural network) genutzt.