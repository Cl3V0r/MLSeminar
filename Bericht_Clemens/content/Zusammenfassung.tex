\chapter{Zusammenfassung und Ausblick}
Die eingangs gestellte Fragestellung, ob sich Fake News von Real News mittels 
Methoden des maschinellen Lernen unterscheiden lassen, kann zumindest eingeschränkt
auf den verwendeten Datensatz beantwortet werden. Die gute Perfomance des Netzes spricht 
dafür, dass es prinzipiell möglich ist einen Teil zu unterscheiden, es bleibt aber ein 
Rest, vorallem von als Real News klassifierten Fake News mit Anteil von $0.14$, bei 
denen die gegebenen Informationen nicht ausreichen, wie in Abbildung \ref{fig:cnfn_hist}
zu sehen ist. Eventuell könnten aktuellere Methoden, wie ein LSTM (Long-Short-Term-Memory)
oder Googles \textsc{Bert}(Bidirectional Encoder Representations from Transformers) bessere
Ergebnisse liefern. Erste Versuche in diese Richtung konnten dies jedoch nicht bestätigen.

Die begrenzte Verwendbarkeit dieses Datensatzes zur Klassifizierung zukünftiger Nachrichten  
werden vorallem durch die Wörter deutlich, die sich auf einen bestimmten Zeitraum beziehen.
Außerdem sind die Themen des Datensatzes auch inhaltlich durch die Themen des US-Wahlkampfes 
dominiert. Zur Verringerung des so begangenen systematischen Fehlers wird es nötig sein,
den Datensatz auf größere Zeiträume auszuweiten. Zusätzlich müsste die Liste der Quellen
stetig aktuell gehalten und manuell erweitert werden, um neue Quellen für Fake News 
nicht auszuschließen.