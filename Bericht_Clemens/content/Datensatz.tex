\section{Charakterisierung und Vorbereitung des Datensatzes}
Der verwendete Datensatz \cite{real_data} enthält vorallem englischsprachige Nachrichten aus dem Jahr 
2016 der Monate Oktober bis Dezember, also vorallem aus der Zeit der US-Präsidentschaftswahl.
Die Eigenschaften der Daten sind dabei der Titel, Text und Herausgeber und werden in die Kategorien 
Real News und Fake News eingeteilt. Die Fake News Daten wurden dabei dem 
\enquote{BS-Detector} \cite{BS} entnommen. Dieser ist eine Erweiterung für den Browser 
der eine Website nach URL-Verlinkungen durchsucht, welche in einer manuell geführten 
Liste unsicherer Quellen auftauchen. Solche unsicheren Quellen veröffentlichen dabei unter anderen 
Fake News, Satire, Verschwörungstheorien oder pseudowissenschaftliche Artikel. Artikel dieser Quellen 
sind dann alle zu einem Fake News Datensatz zusammengefasst worden \cite{fake_data}. Die 
Real News des Datensatzes entstammen dabei zehn vertrauenswürdigen Berichterstattern, wie der 
Washington Post oder der New York Times.